Notre analyse de besoins a fait ressortir la préoccupation et la difficulté
pour les étudiants de faire le bon choix de restaurant à chaque repas.
Ce besoin peut paraitre accessoire ou futile mais l'attente aux restaurants
peut être une véritable source de stress pour un étudiant pressé. Il en
va de même pour la constitution du repas qui peut décevoir l'étudiant une
fois la longue attente terminée. Cela peut conduire certains étudiants à 
élaborer des stratégies complexes pour manger à la bonne heure et au bon
endroit.

Nous avons donc décidé de travailler sur un outil permettant aux étudiants
d'avoir tout les éléments entre les mains au moment de la prise de décision.

Nous avons ensuite identifié les éléments importants utilisés par les étudiants lors de la prise de décision :
\begin{itemize}
\item le temps d'attente avant de pouvoir manger,
\item le menu,
\item la distance entre la localisation actuelle et le restaurant.
\end{itemize}

Les étudiants étrangers connaissant peu le campus peuvent en plus avoir besoin
d'un itinéraire guidé pour se rendre aux restaurants.

Ce sont ces fonctionnalités principales qui sont à la base de la raison
d'être de l'application.
Ces fonctionnalités principales donnent lieu aux fonctionnalités
détaillées suivantes :
\begin{itemize}
\item visualisation du temps d'attente en fonction de
l'heure,
\item visualisation du menu du jour,
\item visualisation de la localisation sur un plan du campus,
\item tri des restaurants selon le temps d'attente en fonction de l'heure,
\item tri des restaurants selon la distance,
\item envoie d'une notification concernant le meilleur choix de restaurant
(selon les critères de tri ci-dessus).
\end{itemize}

De plus l'application doit s'adapter à l'utilisateur et doit prendre en
compte : 
\begin{itemize}
\item les restaurants fréquentés par l'utilisateur,
\item le critère de classement préféré de l'utilisateur,
\item si l'utilisateur désire être notifié,
\item à quelle heure l'utilisateur désire être notifié,
\item quels jours l'utilisateur désire être notifié.
\end{itemize}