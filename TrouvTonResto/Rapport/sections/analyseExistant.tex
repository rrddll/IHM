Pour l’instant, peu de solutions existent pour faciliter la vie des 
utilisateurs étudiants ou personnels sur le campus de la Doua dans leur 
choix de restaurant. Nous pouvons répertorier les services existants suivants.

Pour les étudiants de l'INSA, sur le site planete.insa-lyon.fr sont disponibles 
le menu du jour des restaurants de l'INSA ainsi que le crédit repas de 
l'abonné. Ce service propose notamment quelques problèmes : 
\begin{itemize}
\item la connexion depuis un smartphone est compliquée : le site n'est pas 
adapté aux petits écrans,
\item le menu est téléchargeable en format .doc (MS Word) uniquement, ce qui 
n'est pas pratique pour les utilisateurs,
\item le menu présent est rarement à jour et ne correspond pas toujours au 
menu réellement proposé dans le restaurant,
\item il est possible de visualiser son crédit repas mais il est impossible 
de le recharger.
\end{itemize}

En clair, ce dispositif n'est disponible que pour les étudiants de l'INSA 
et ne propose pas un service complet.

Le CROUS a également mis un place un service qui peut se rapprocher de notre 
sujet : un application smartphone (IZLY) est disponible et permet de 
recharger son crédit repas. Cependant cette solution reste limitée à ce 
problème de rechargement du crédit repas, et ne permet pas aux utilisateurs 
de choisir leur restaurant, et ne leur donne aucune information spécifique 
aux restaurants du campus.

Au final, aucun dispositif existant ne répond réellement à la problématique 
que nous nous sommes fixés, qui est de créer une application qui permet à 
un utilisateur du campus de la Doua de choisir facilement dans quel restaurant 
il va manger chaque midi.
