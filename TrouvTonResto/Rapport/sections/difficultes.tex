\subsection{Difficultés techniques rencontrées}
Le développement sous Android étant nouveau pour beaucoup d'entre nous et la
plateforme de développement n'étant pas disponible au début du projet sur les
machines du département, il nous a fallu commencer par installer tout
l'environnement de travail. Nous avons perdu beaucoup de temps au début
pour que chacun installe Android Studio, les SDKs nécessaires et configure
l'émulateur. \\

Voulant insérer une carte Google dans notre application et voulant des
transitions animées entre nos différentes fenêtres, nous avons été confronté
à des problèmes de compatibilité entre les versions d'Android. En effet, 
certaines fonctionnalités que nous voulions utiliser n'étaient pas disponible
avec la version minimale du SDK que nous avions choisie au départ. C'est pourquoi
nous avons fini par prendre une version du SDK plus récente (la 17). Certes,
moins d'utilisateurs pourraient utiliser notre application mais comme l'objectif
du projet était de créer un prototype, nous avons décidé de mettre le focus sur 
la rapidité de développement plutôt que sur la comptabilité. \\

Sur la fenêtre avec les détails d'un restaurant, il y a un graphique dynamique
qui permet à l'utilisateur de connaître rapidement une estimation du temps
d'attente. Pour cela, il doit pouvoir déplacer avec le doigt
la barre verticale rouge indiquant l'heure. Néanmoins, n'ayant
pas le temps de développer tout un système de graphique dynamique, nous avons
préféré afficher seulement une image. Celle-ci permet d'avoir un bon aperçu du
visuel final mais ne permet pas de tester les interactions avec le graphique.\\

Une dernière difficulté technique que nous avons rencontrée est due au fait que
notre application est un prototype. Toutes nos données sont factices et stockées
en dur dans l'application. Du coup, il n'y aucune communication avec un serveur
distant, aucune notification, aucun calcul, etc. Néanmoins, nous voulions avoir
une application le plus proche possible de la réalité. C'est pourquoi nous avons
utilisé des minuteurs pour simuler les temps d'attentes et l'arrivée des 
notifications. 

\subsection{Problèmes d'utilisabilités rencontrés}

Le principal problème d'utilisabilité que nous avons rencontré se trouve sur 
la fenêtre d'affichage des détails d'un restaurant. Nous voulions pouvoir 
permettre à un utilisateur avancé de changer plus rapidement de restaurant
en faisant un simple glissement du doigt, comme il changerait de discussion dans sa
messagerie. Néanmoins, nous voulions également trouver un moyen simple pour
qu'un utilisateur puisse découvrir cette fonctionnalité. Cela nous a amené 
à nous poser pleins de questions :  

\begin{itemize}
\item Etait-ce une pratique courante d'Android qui ne nécessite pas d'apprentissage ?
\item Devrions-nous afficher des flèches pour expliquer à un utilisateur qu'il pouvait
naviguer entre les restaurants ? 
\item Et si nous utilisions ces flèches, où fallait-elles les mettre ?
\item Au milieu, de part et d'autre de l'écran ?
\item De chaque côté du titre "Temps d'attente" ?
\item Ou bien au dessus de ce titre ? 
\end{itemize}

Nous avons fini par implémenter cette dernière solution car selon-nous il est nécessaire
de montrer à l'utilisateur ce qu'il peut faire et de ne pas lui cacher des fonctionnalités.
Pour la position des flèches, de part et d'autre de l'écran nous perdions trop de place horizontale, 
place qui est réduite sur les téléphones. Les mettre de chaque côté du titre amenait une
incohérence. Nous avions l'impression de changer de graphique et non de restaurant. C'est
pourquoi nous avons décidé de rajouter une barre au dessus de ce titre pour y mettre les
flèches. 
\\
 
Un autre problème que nous avons rencontré sur cette page est la taille des menus. Si 
nous prenions, par exemple, les entrées du Restaurant universitaire de la Doua, chaque midi
il y en a une quinzaine, obligeant de ce fait l'utilisateur
à faire défiler l'écran vers le bas pour afficher ce qui est réellement intéressant, le 
plat principal. C'est pourquoi, nous avons décidé de mettre le menu dans une liste extensible.
Ainsi, l'utilisateur ne voit au départ que les grandes catégories du menu (entrée, plat et
dessert). C'est seulement en cliquant sur ce qui l'intéresse qu'il pourra voir son contenu.
Cela permet d'éviter de surcharger cet écran. 

\unscaledFigure{images/resto.png}{Détail d'un restaurant où nous pouvons voir 
les flèches pour changer de restaurant et la liste extensible}
 
 
Sur la page listant les restaurants, il existe un curseur qui permet à l'utilisateur
d'avoir des prévisions pour l'heure de son choix. Néanmoins, le calcul des
prévisions prend un certain temps. Durant le calcul, il ne faut pas que l'utilisateur 
puisse cliquer sur un restaurant alors que les données ne celui-ci ne sont pas
encore à jour. Comment prévenir l'utilisateur qu'un calcul est en cours et qu'il 
faut attendre ? Pour répondre à cette question, nous avons décidé de griser la 
liste des restaurants à partir du moment où le curseur est déplacé. 
\unscaledFigure{images/restos_slide.png}{Liste des restaurants grisée car l'utilisateur déplace le curseur}

Évidemment notre application contient un écran de préférences pour permettre
à l'utilisateur de modifier les réglages de l'application. Néanmoins, nous 
voulions que dès la première utilisation l'application soit personnalisée. A 
quoi sert d'afficher un restaurant dans lequel l'utilisateur ne puisse pas aller ?
Un utilisateur lambda a rarement le réflexe d'aller modifier les préférences de
son application. C'est pourquoi nous avons décidé d'afficher un écran de première
ouverture qui permet à l'utilisateur de personnaliser l'application et de lui
expliquer brièvement son fonctionnement.
\unscaledFigure{images/first_screen_1.png}{Un des écrans affichés à la première connexion qui explique brièvement le fonctionnement de l'application}

Le dernier problème d'utilisabilité que nous avons rencontré se situe
dans l'affichage de la carte. Au début, nous centrions la carte sur la 
position de l'utilisateur comme cela se fait dans beaucoup d'applications. 
Néanmoins, si l'utilisateur ne se trouve pas sur le campus mais dans ses
alentours, il devra faire glisser la carte pour chercher les restaurants. 
Nous avons trouvé que ceci n'était pas très pertinent car, quelque soit la
position de l'utilisateur, il ne sera affiché que les restaurants du campus. 
C'est pourquoi nous avons fait un système qui centre la carte sur les restaurants
et non sur la position GPS du téléphone. 
  