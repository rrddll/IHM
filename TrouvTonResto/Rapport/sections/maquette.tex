Nous avons commencé par tester des idées d'IHM à l'aide de post-its
représentant les enchaînements d'écrans. Cette façon de faire nous a
permis d'éviter certaines organisations d'écran auxquelles nous avons
pensé qui se sont révélées peu ergonomiques lors des tests. Un exemple
d'agencement que cette expérimentation nous a fait choisir est celle du
bouton "Itinéraire" qui a été fixé au lieu de se retrouver en bas d'un
panneau déroulant. En effet, nous avions simulé le scroll avec le bout
de post-it présenté en figure \ref{bout} et en ayant le bouton fixé
(figure \ref{fixe}). Il nous a finalement paru plus judicieux de laisser ce
bouton accessible, même si le menu du restaurant venait à occuper de la
place. Les maquettes réalisées sont présentées ci-après\footnote{Les
scan des post-it n'ayant pas donné de grands résultats (du fait sûrement
de la fluorescence de ceux-ci, nous avons dû les retravailler, ce qui
explique l'aspect granuleux des images.}.


\unscaledFigure[!h]{images/maquettes/notifications.jpg}{Notification
indiquant à l'utilisateur le meilleur restaurant pour aller manger}
\unscaledFigure[p]{images/maquettes/accueil.jpg}{Écran d'accueil de
l'application. C'est aussi sur cet écran qu'on tombe en ayant cliqué sur
la notification.}
\unscaledFigure[p]{images/maquettes/details.jpg}{Détails sur un
restaurant. On y voit le graphe du temps d'attente ainsi que le menu. On
a la possibilité de calculer un itinéraire.\label{fixe}}
\unscaledFigure[p]{images/maquettes/boutMenu.jpg}{Bout de post-it nous
ayant servi à tester un menu scrollable\label{bout}}
\unscaledFigure[p]{images/maquettes/itineraire.jpg}{Calcul de
l'itinéraire par l'application maps}
\unscaledFigure[p]{images/maquettes/menu.jpg}{Menu permettant d'accéder
à la liste des restaurants, à la carte ainsi qu'au paramétrage de
l'application.}
\unscaledFigure[p]{images/maquettes/carte.jpg}{Vue du plan avec
géolocalisation des restaurants.}
\unscaledFigure[p]{images/maquettes/cartedetail.jpg}{Affichage du détail
d'un restaurant sur la carte.}
