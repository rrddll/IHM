A chaque lancement de l'application vous trouverez l'écran de la figure \ref{loadingg} qui 
permet à l'application de récupérer les informations sur le réseau.
\unscaledFigure{images/loading.png}{Ecran de chargement de 
l'application \label{loadingg}}

Au premier lancement de l’application, un certain nombre de questions vous
seront posées ayant pour objectif de définir votre profil utilisateur. Les 
réponses apportées seront alors mémorisées par l’application afin de
répondre au mieux à vos attentes. L'écran \ref{first} représente la première question
qui vous sera posée. L'écran \ref{second} représente la seconde question qui vous 
sera posée, tant dis que l'écran \ref{third} représente la troisième question qui vous
sera posée. 
\unscaledFigure{images/first_screen_1.png}{Premier écran lors du premier
lancement de l'application \label{first}}
\unscaledFigure{images/first_screen_2.png}{Second écran lors du premier
lancement de l'application \label{second}}
\unscaledFigure{images/first_screen_3.png}{Troisème écran lors du premier
lancement de l'application \label{third}}

Ces paramètres peuvent être modifiés à tout moment dans la menu paramètre
accessible via le volet de navigation comme le présente l’image ci-après : 
\unscaledFigure{images/drawer.png}{Volet de navigation}

Dans les paramètres, il est aussi possible d' activer/supprimer la notification
journalière automatique qui vous est envoyée systématiquement à midi tous les jours.
Il est également possible de changer l’heure de cette notification afin qu’elle 
puisse vous prévenir au moment le plus opportun selon votre situation. 
Enfin, il est possible, comme le montre l'écran \ref{prefs} de choisir si vous 
préférez que vos restaurants soient choisis selon la proximité ou le temps 
d'attente.
\unscaledFigure{images/prefs.png}{Ecran des paramètres \label{prefs}}

Une fois votre application paramétrée avec vos préférences, vous arriverez
systématiquement sur l'écran \ref{restos} lors d’un nouveau lancement. Nous
rapellons que toute application sous android peut tourner en tâche de fond,
n’oubliez pas de la quitter intégralement lorsque vous n’en avez plus l’utilité.
Cela augmentera l’autonomie de votre appareil et permettra de limiter vos
consommations de datas. 
\unscaledFigure{images/restos.png}{Ecran principal de l'application 
\label{restos}}

Sur l'écran \ref{slider} sont rangés par ordre correspondant à vos paramètres l’ensemble des
restaurants cochés précédement. Il est possible de régler l’heure à laquelle
vous souhaitez déjeuner grâce au slider situé en bas de l’écran.
\unscaledFigure{images/restos_slide.png}{Le slider permet d'ajuster l'heure
à laquelle vous souhaitez déjeuner \label{slider}}
 
Il est possible sur un simple clique sur un restaurant de l’écran précédent de
visualiser les informations précise de ce dernier. Voici un exemple avec le
restaurant universitaire de la Doua : 
\unscaledFigure{images/resto.png}{Les détails du restaurant universitaire de
la Doua}

Sur cet écran, bon nombre d’informations sont disponibles. En premier lieu,
vous trouverez une estimation du temps d’attente en fonction de l’heure de
passage au restaurant. Cette estimation future est basée sur les connaissances
de l’application. Cependant, l’heure affichée avec la barre de séparation est le
temps réel d’attente, défini par des capteurs placés judicieusement sur toutes
les files d’attente des différents points de restauration. Voici un exemple d'un
exemple de graphique de fil d'attente de l'Olivier :
\unscaledFigure{images/graph_attente.png}{Les détails du restaurant  l'Olivier}

Sur le même écran, légèrement plus bas, se trouve le détail des entrées,
plats et desserts qui sont actuellement disponibles dans le restaurant
sélectionné. Dans l’écran \ref{detailsMenu}, nous sommes toujours sur
notre écran de l'Olivier et nous pouvons voir qu’il y aura en entée : 
de l'oeuf mimosa, de la quiche à la volaille et du saucisson brioché, en plat : 
de la brochette servis avec des frites, du poisson servi avec des légumes et de 
la pizza et en dessert : seulement des beignets sont proposés.  
\unscaledFigure{images/menus_restaurant_details.png}{Les détails des menus
de l'Olivier \label{detailsMenu}}

Enfin, en dessous des plats disponibles dans le restaurant se trouve le
bouton "Y aller !" permettant de démarrer l’application Maps (Google) afin de
vous rendre de votre lieu actuel vers le restaurant sélectionné. Voici un exemple
lorsque l'on appuie sur le bouton "Y aller" :
\unscaledFigure{images/google_maps.png}{Utilisation de l'application Google Map
afin de réaliser les déplacements}

Depuis le détail d’un restaurant il vous est possible de swipper des 
détails d’un restaurant à un autre facilement en glissant son doigt vers la
gauche ou la droite de l’écran. Les défilements ont lieu d'un restaurant à l'autre
conformément à la liste triée que nous avons vu précédement (cf écran \ref{restos}).

Enfin, pour les nouveaux étudiants, il est possible d'accéder via le volet de navigation
à la carte du campus afin de localiser rapidemment où sont situés les différents
restaurants présents dans l'application.
\unscaledFigure{images/map.png}{Carte du campus avec la localisation des différents
restaurants de l'application}

Pour toute information complémentaire, l’héxanome H4404 reste disponible
à l’adresse mail suivante : heptanhommes@googlegroups.com. 