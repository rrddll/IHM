Au premier lancement de l’application, un certain nombre de questions vous
seront posées ayant pour objectif de définir votre profil utilisateur. Les 
réponses apportées seront alors mémorisées par l’application afin de
répondre au mieux à vos attentes. Vous trouverez ci après l’écran de
chargement de l’application (image 1), ainsi qu’un aperçu de la première
et de la seconde question qui vous sera poser au premier lancement de
l’application (image 2 et 3) : 

[TODO : insert image chargement + 2 screenshots des questions de la 1ere connexion]

Ces paramètres peuvent être modifiés à tout moment dans la menu paramètre
accessible via le volet de navigation comme le présente l’image ci-après (image 4) : 

[TODO : insert image menu + cercle sur le menu paramètres]

Dans les paramètres (image 5), il est aussi possible de supprimer la notification
automatique qui vous sera envoyé systématiquement à midi tous les jours. Il est
également possible de changer l’heure de cette notification afin qu’elle puisse vous
prévenir au moment le plus opportun selon votre situation. 

[TODO : insert image paramètres] 

Une fois votre application paramétrée avec vos préférences, vous arriverez
systématiquement sur cet écran lors d’un nouveau lancement (image 6). Nous
rapellons que toute application sous android peut tourner en tache de fond,
n’oubliez pas de la quitter intégralement lorsque vous n’en avez plus l’utilité.
Cela augmentera l’autonomie de votre appareil et permettra de limiter vos
consommations de datas. 

[TODO : insert écran principal]

Sur cet écran sont rangés par ordre d’attente minimum l’ensemble des
restaurants cochés précédement. Il est possible de régler l’heure à laquelle
vous souhaitez déjeuner grâce au slider situé en bas de l’écran (image 7)

[TODO : insert écran principal comme précédemment avec zoom sur le slider]

Il est possible sur un simple clique sur un restaurant de l’écran précédent de
visualiser les informations précise de ce dernier. Voici un exemple avec le
restaurant Castor et Pollux (image 8) : 

[TODO : insert image Castor et Pollux]

Sur cet écran, bon nombre d’informations sont disponibles. En premier lieu,
vous trouverez une estimation du temps d’attente en fonction de l’heure de
passage au restaurant. Cette estimation future est basée sur les connaissances
de l’application. Cependant, l’heure affichée avec la barre de séparation est le
temps réel d’attente, défini par des capteurs placés judicieusement sur toutes
les files d’attente des différents points de restauration (image 9) : 

[TODO : insert image avec zoom sur le graphique des heures]

Sur le même écran, légèrement au dessous se trouve le détail des entrées,
plats et desserts qui sont actuellement disponibles dans le restaurant
sélectionné. Dans l’écran ci-après (image 10), nous sommes toujours sur
notre écran Castor et Pollux et nous pouvons voir qu’il y aura en entée : 
X, Y et Z, en plat : T, V, W et en dessert : Alpha, Beta, VariableSansNom. 

[TODO : insert image détail menu ouvert]

Enfin, en dessous des plats disponibles dans le restaurant se trouve un
bouton permettant de démarrer l’application Maps (application Google) afin de
vous rendre de votre lieu actuel vers le restaurant sélectionné. Voici un exemple
de cette carte (image 11) :

[TODO : insert le bouton + la map]  

Enfin, depuis le détail d’un restaurant il vous est possible de swipper des 
détails d’un restaurant à un autre facilement en glissant son doigt vers la
gauche ou la droite de l’écran (image 12) : 

[TODO : insert detail restaurant + fleche pour montrer le swipe]

Pour toute information complémentaire, l’héxanome H4404 reste disponible
à l’adresse mail suivante : heptanhommes@googlegroups.com. 