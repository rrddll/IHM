\subsection{Première ouverture de l'application}
Au premier lancement de l'application, un certain nombre de questions vous
seront posées ayant pour objectif de définir votre profil utilisateur. Les 
réponses apportées seront alors mémorisées par l'application afin de
répondre au mieux à vos attentes. Nous vous conseillons de répondre à ces 
questions mais si vous préférez accéder directement au c\oe ur de l'application, 
cliquez sur \includegraphics{images/ignorer.png}. 

Pour accéder aux autres questions, cliquez sur le bouton \includegraphics{images/next.png}
ou glissez votre doigt sur l'écran de la droite vers la gauche comme sur 
l'image ci-dessous :
\strechedFigure{images/doc_first_screen.png}{Changement d'écran lors du premier lancement \label{first}}

Une fois toutes les étapes complétées, cliquez sur \includegraphics{images/ok.png} pour 
accéder à l'application. 

\subsection{Visualiser la liste des restaurants}
Lors du lancement de l'application, vous arriverez sur l'écran principal qui
liste vos restaurants préférés. Sur cet écran, les restaurants sont rangés selon l'ordre défini par vos paramètres. Nous
rappelons que toute application sous Android peut tourner en tâche de fond.
N'oubliez donc pas de la quitter intégralement lorsque vous n'en avez plus l'utilité.
Cela augmentera l'autonomie de votre appareil et permettra de limiter votre
consommation de données. 
Voici un aperçu de l'écran principal :
\unscaledFigure{images/restos.png}{Écran principal de l'application 
\label{restos}}

Il est possible de régler l'heure à laquelle
les estimations affichées à droite sont calculées en déplaçant le curseur situé
en bas de l'écran : 
\unscaledFigure{images/slider.png}{Ce curseur permet d'ajuster l'heure
à laquelle vous souhaitez obtenir les estimations \label{slider}}
 
\subsection{Visualiser les détails d'un restaurant}
Pour afficher les détails d'un restaurant, cliquez sur le nom de celui-ci dans
la liste des restaurants :  
\strechedFigure{images/doc_resto.png}{Accès aux détails d'un restaurant}

Sur cet écran, un bon nombre d'informations sont disponibles. En premier lieu,
vous trouverez une estimation du temps d'attente en fonction de l'heure de
passage au restaurant. Cette estimation est basée sur les connaissances
de l'application. L'heure affichée à coté du curseur rouge est par défaut le
temps d'attente réel actuellement. Vous pouvez obtenir une estimation du temps
d'attente en le faisant glisser avec votre doigt vers la droite.
Le temps d'attente réel est défini par des capteurs placés judicieusement
sur toutes les files d'attente des différents points de restauration.

Pour afficher les détails du menu, cliquez sur la catégorie qui vous intéresse 
(Entrée, Plat et Dessert) en bas du graphique. En cliquant, la liste des entrées, 
plats ou desserts vont s'afficher. 
Sur le même écran, légèrement plus bas, se trouve le détail des entrées,
des plats et des desserts qui sont actuellement disponibles dans le restaurant
sélectionné. Comme nous pouvons le voir dans l'exemple suivant, après
avoir cliquez sur Entrée nous pouvons voir qu'il y a de l'\oe uf mimosa, 
de la quiche à la volaille et du saucisson brioché en ce moment.
\unscaledFigure{images/menus_restaurant_details.png}{Les détails des menus
de l'Olivier \label{detailsMenu}}

Pour savoir comment aller à ce restaurant depuis votre position actuelle, cliquez
sur le bouton "Y aller !" en bas de l'écran. 
Ceci va démarrer l'application Google Maps afin de vous guider pour
vous rendre de votre lieu actuel vers le restaurant.

Depuis le détail d'un restaurant vous pouvez afficher rapidement les détails 
d'un autre restaurant en glissant votre doigt vers la
gauche ou la droite de l'écran. Vous pouvez également utiliser les flèches présentent
de part et d'autre de titre "Détail". 
\strechedFigure{images/doc_resto_switch.png}{Passer des détails d'un restaurant à un autre}
Les changements ont lieu d'un restaurant à l'autre
conformément à la liste triée que nous avons vu précédemment (cf écran \ref{restos}).

\subsection{Trouver les restaurants sur le campus}
Pour les nouveaux étudiants, il est possible d'accéder via le volet de navigation
à la carte du campus afin de localiser rapidement où sont situés les différents
restaurants présents sur le campus.
Pour cela, cliquez sur \includegraphics{images/drawerButton.png}. Ce bouton va ouvrir le volet de navigation. Cliquez ensuite sur "Carte"
pour accéder à la fenêtre ci-contre : 
\unscaledFigure{images/map.png}{Carte du campus avec la localisation des
différents restaurants présents}

En cliquant sur un des marqueurs rouges, un popup apparaît avec le nom du
restaurant. Par un simple clic sur ce popup, vous pouvez accéder à la fenêtre
détaillant un restaurant présentés ci-dessus.

\subsection{Personnaliser l'application}
Les paramètres de l'application peuvent être modifiés à tout moment dans la menu
préférences. Pour y accéder cliquez sur \includegraphics{images/drawerButton.png}.
Ce bouton va ouvrir le volet de navigation. Cliquez ensuite sur "Préférences"
pour accéder à la fenêtre ci-contre : 
\unscaledFigure{images/prefs.png}{Fenêtre des préférences}

Dans la section "Critère de classement des restaurants", il est possible de choisir si vous 
préférez que vos restaurants soient choisis et triées selon la proximité ou le temps 
d'attente.

Rajouter un truc sur les restaurants préférés en fonction de ce que julien va modifier
cet aprèm

La notification quotidienne qui vous est envoyée par défaut à midi chaque 
jour de la semaine. Néanmoins, il est possible d'activer/désactiver  en cliquant sur 
\includegraphics{images/onoff.png} en face de "Recevoir des notifications". 
Il est également possible de changer l'heure de cette notification
afin qu'elle puisse vous prévenir au moment le plus opportun selon votre situation. 
Si les notifications sont activées, un sélecteur d'heure va s'afficher en dessous comme 
sur l'image ci-contre : \unscaledFigure{images/time.png}{Outil permettant de 
sélectionner l'heure d'envoi de la notification}


Pour toute information complémentaire, l'heptanôme H4404 reste disponible
à l'adresse email suivante : heptanhommes@googlegroups.com. 