Au premier lancement de l’application, un certain nombre de questions vous
seront posées ayant pour objectif de définir votre profil utilisateur. Les 
réponses apportées seront alors mémorisées par l’application afin de
répondre au mieux à vos attentes. Vous trouverez ci après l’écran de
chargement de l’application, ainsi qu’un aperçu de la première
et de la seconde question qui vous sera poser au premier lancement de
l’application (image 2 et 3) : 
\unscaledFigure{images/first_screen_1.png}{Premier écran lors du premier
lancement de l'application}
\unscaledFigure{images/first_screen_2.png}{Second écran lors du premier
lancement de l'application}
\unscaledFigure{images/first_screen_3.png}{Troisème écran lors du premier
lancement de l'application}

Ces paramètres peuvent être modifiés à tout moment dans la menu paramètre
accessible via le volet de navigation comme le présente l’image ci-après : 
\unscaledFigure{images/drawer.png}{Volet de navigation}

Dans les paramètres, il est aussi possible de supprimer la notification
automatique qui vous sera envoyé systématiquement à midi tous les jours. Il est
également possible de changer l’heure de cette notification afin qu’elle puisse vous
prévenir au moment le plus opportun selon votre situation. 
\unscaledFigure{images/prefs.png}{Ecran des paramètres}

Une fois votre application paramétrée avec vos préférences, vous arriverez
systématiquement sur cet écran lors d’un nouveau lancement. Nous
rapellons que toute application sous android peut tourner en tache de fond,
n’oubliez pas de la quitter intégralement lorsque vous n’en avez plus l’utilité.
Cela augmentera l’autonomie de votre appareil et permettra de limiter vos
consommations de datas. 
\unscaledFigure{images/restos.png}{Ecran principal de l'application}

Sur cet écran sont rangés par ordre d’attente minimum l’ensemble des
restaurants cochés précédement. Il est possible de régler l’heure à laquelle
vous souhaitez déjeuner grâce au slider situé en bas de l’écran
\unscaledFigure{images/restos_slide.png}{Le slider permet d'ajuster l'heure
à laquelle vous souhaitez déjeuner}

Il est possible sur un simple clique sur un restaurant de l’écran précédent de
visualiser les informations précise de ce dernier. Voici un exemple avec le
restaurant universitaire de la Doua : 
\unscaledFigure{images//resto.png}{Les détails du restaurant universitaire de
la Doua}

Sur cet écran, bon nombre d’informations sont disponibles. En premier lieu,
vous trouverez une estimation du temps d’attente en fonction de l’heure de
passage au restaurant. Cette estimation future est basée sur les connaissances
de l’application. Cependant, l’heure affichée avec la barre de séparation est le
temps réel d’attente, défini par des capteurs placés judicieusement sur toutes
les files d’attente des différents points de restauration : 
\unscaledFigure{images//resto.png}{Les détails du restaurant universitaire de
la Doua}

Sur le même écran, légèrement au dessous se trouve le détail des entrées,
plats et desserts qui sont actuellement disponibles dans le restaurant
sélectionné. Dans l’écran ci-après, nous sommes toujours sur
notre écran Castor et Pollux et nous pouvons voir qu’il y aura en entée : 
X, Y et Z, en plat : T, V, W et en dessert : Alpha, Beta, VariableSansNom. 
\unscaledFigure{images/resto.png}{Les détails du restaurant universitaire de
la Doua}

Enfin, en dessous des plats disponibles dans le restaurant se trouve le
bouton "Y aller !" permettant de démarrer l’application Maps (Google) afin de
vous rendre de votre lieu actuel vers le restaurant sélectionné. Voici un exemple
de cette carte :
\unscaledFigure{images//resto.png}{Les détails du restaurant universitaire de
la Doua}

Depuis le détail d’un restaurant il vous est possible de swipper des 
détails d’un restaurant à un autre facilement en glissant son doigt vers la
gauche ou la droite de l’écran (image 12) : 
\unscaledFigure{images//resto.png}{Les détails du restaurant universitaire de
la Doua}

Enfin, pour les nouveaux étudiants, il est possible d'accéder via le volet de navigation
à la carte du campus afin de localiser rapidemment où sont situés les différents
restaurants présents dans l'application.
\unscaledFigure{images/map.png}{Carte du campus avec la localisation des différents
restaurants de l'application}

Pour toute information complémentaire, l’héxanome H4404 reste disponible
à l’adresse mail suivante : heptanhommes@googlegroups.com. 