\subsection{Méthode retenue}

L'analyse des besoins devait être réalisée au préalable de toute conception ou
développement de l'application. Nous avons choisi d'effectuer notre enquête en
établissant des interviews avec un questionnaire prédéfini et construit préalablement 
avec l'ensemble de l'équipe. Nous avons préféré cette méthode qui 
vise à directement rencontrer les utilisateurs plutôt qu'une version numérique
du questionnaire pour une raison simple. Nous avions pour crainte que les 
questionnaires en ligne ne soient complétés uniquement 
par des élèves de 3 ou 4ème année en informatique. Ceci aurait rendu l'échantillon
peu représentatif et aurait ainsi faussé la pertinence des résultats obtenus. \\
Nous nous sommes donc séparés en deux groupes de deux et un groupe de trois personnes.
Notre enquête a été réalisé dans différents lieux stratégiques : 
\begin{itemize}
\item devant le Grillon, le Castor gourmand et l'Olivier, lieu où il n'y a pas
moins de trois restaurants distincts dont deux insaliens,
\item devant le Prévert et le Castor et Pollux, endroit très fréquenté le midi,
\item devant le Restaurant Universitaire de la Doua (\textsc{ru}), restaurant affectionné
par beaucoup d'étudiants et pas seulement des insaliens,
\item et un groupe a également fait un tour dans les laboratoires du \textsc{citi}, où
nous retrouvons une nouvelle fois des profils différents (professeurs et 
doctorants).
\end{itemize}

\subsection{Contenu du questionnaire}
Le questionnaire a été tourné d'une façon spécifique. Nous avons dans un 
premier temps posé des questions générales pour obtenir le profil de notre
futur utilisateur, sans préciser le but du sondage. Puis nous posions 
des questions générales sur leurs pratiques concernant les repas du midi
en procédant à des mises en situation simple. Enfin, nous dévoilions le
sujet du sondage et les fonctionnalités que nous envisagions de
développer pour savoir si oui ou non l'utilisateur y voyait un intérêt. 
Évidement, nous étions ouverts à toutes les idées qu'il pouvait avoir
afin de découvrir éventuellement de nouvelles fonctionnalités. \\
Voici quelques exemples de questions présentes dans le questionnaire. 
Il est évident que les questions différaient d'un interview à 
l'autre en fonction de l'implication et des réponses de l'interviewé. 
\begin{itemize}
\item En quelle année d'étude es-tu ? Es-tu insalien ?
\item Où manges-tu les midis ?
\item Pourquoi ne manges-tu pas ailleurs ? (Il fallait essayer d'aborder les points
suivants : le temps d'attente, le prix, la facilité de manger avec des amis,
la proximité, la qualité, la variété des menus et le menu du jour)
\item Où manges-tu si tu as moins d'une heure pour manger ?
\item Si tu avais su qu'un autre restaurant avait un temps d'attente inférieur, 
cela aurait-il influencé ton choix ? 
\item Si tu arrives dans le restaurant et que le menu ne te plaît pas,
que fais-tu ?
\item Si tu avais connu le menu au préalable, aurais-tu changé ton choix ?
\item Serais-tu intéressé par des notifications sur ton mobile afin de te dire 
quel restaurant correspond le mieux à tes critères ? (sachant que des 
critères tels que le prix, le temps d'attente, la proximité le menu ou autres
pourraient être pris en compte)
\item Quelles fonctionnalités supplémentaires t'intéresseraient d'avoir sur 
une application pour ton smartphone ?
\item Serais-tu intéressé par la possibilité de recharger ton forfait sur ton smartphone ?
\end{itemize}

\subsection{Échantillon de la population}
Suite à l'élaboration de ces questionnaires, nous avons recensé 
les résultats dans un tableur et discuté des différentes réponses en 
fonction du profil de notre population. Nous avons obtenu 27 réponses. 
Globalement, nous avons beaucoup d'insaliens. Ils sont néanmoins équitablement répartis
entre le premier et le second cycle. Parmi ces étudiants, deux
étudiants d'échange sont représentés. Ils avaient des réponses bien
différentes des autres ce qui nous a permis de distinguer très clairement un nouveau profil
utilisateur. Nous avons également un étudiant de l'université Lyon 1,
un doctorant et deux professeurs du département \textsc{tc}. 

\subsection{Résultats obtenus}
Dans la quasi totalité des cas, à l'exception des gens moins pressés 
du fait de leurs horaires plus flexibles que ceux de l'usager moyen, le
facteur temps est très important.  Sans cesse la réponse est : j'irais
au plus proche, au plus rapide. Cela
n'est pas choquant car en quelques minutes la file
d'attente peut passer de quelques minutes d'attente à plusieurs
dizaines de minutes. Pour beaucoup d'étudiant, l'argument expliquant cette
importance est que le temps entre midi et quatorze heures permet de faire de
nombreuses choses comme des activités associatives, travailler sur un devoir à rendre,
travailler en projet, faire une réunion de travail ou bien une sieste. Les raisons ne manquent pas 
pour expliquer que tout le monde veuille manger rapidement. 

Certains profils se distinguent tout de même. Par exemple, nous avons interrogé 
une étudiante végétarienne qui est plus intéressée par la qualité de la nourriture.
Un autre exemple de profil distinct se retrouve auprès de nos deux étudiants d'échange interviewés.
Ceux-ci aimeraient savoir la localisation des restaurants sur le campus et éventuellement une
explication pour pouvoir s'y rendre. Ne connaissant pas les différents restaurants, les
différents prix pratiqués ou autre, ces étudiants se retrouvent parfois un peu perdus au
début de l'année et suivent finalement leur groupe ou leurs amis.\\

Quand nous discutions des fonctionnalités propres de l'application, certains
aspects se sont dégagées : 
\begin{itemize}
\item Les étudiants veulent connaître le temps d'attente en temps réel afin de
faire leur choix rapidement en sortant de cours. 
\item Aucun n'a souhaité avoir un contenu social dans cette application qui 
pourrait permettre de manger avec ses amis. Ceux qui essayent de coordonner leur
repas ne le font que rarement le midi (à l'exception des collègues d'une 
même classe) et préfèrent donc se rencontrer le soir. Ceux qui se coordonnent
le midi utilisent des applications déjà existantes telles que les \textsc{sms},
WhatsApp, Messenger, etc.
\item Certains ont demandé de pouvoir voir le menu sur demande avec
éventuellement une image des plats. Cette requête vient des étudiants d'échange
qui parfois ne comprennent pas ce qui se retrouvera dans leur assiette.
\item Les étudiants sont pour l'envoi d'une notification leur disant quel
restaurant correspond le mieux à leur attente pour la journée. 
\item Un affichage du prix est demandé par certains, bien que cela n'est que
peu d'importance pour d'autres étant donné que beaucoup de restaurants coûtent
globalement la même chose. Un étudiant paye le même prix au Grillon, à l'Olivier,
au Prévert et Castor et Pollux. Les prix du Castor Gourmand sont variables mais
restent légèrement supérieurs au prix du \textsc{ru} qui arrivent toujours en tête sur
ce critère.
\end{itemize}

La partie ultérieure présentant la conception contiendra les différents écrans
composant l'application. Nous ferons des références vers cette partie afin
de justifier nos choix conceptuels.
