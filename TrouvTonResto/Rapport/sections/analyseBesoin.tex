\subsection{Méthode retenue}

L’analyse des besoins devait être réalisée au préalable de toute conception ou
développement de l’application. Nous avons choisi d’effectuer notre enquête en
établissant des interviews avec un questionnaire prédéfini et construit avec 
l’ensemble de l’équipe au préalable. Nous avons préféré cette méthode qui 
vise à directement rencontrer les utilisateurs plutôt qu’une version numérique
du questionnaire pour une raison simple : nous avions pour crainte que les 
questionnaires en ligne n’aurait été dans l’ensemble complétés uniquement 
par des élèves de 3 ou 4ème année informatique, ce qui aurait rendu l’échantillon
peu représentatif et aurait ainsi fausser la pertinence des résultats obtenus. \\
Nous nous sommes donc séparés en trois groupes de deux membres du groupe
et un groupe de trois personnes. Notre enquête a été réalisé dans différents 
lieux stratégiques : 
\begin{itemize}
\item Devant le grillon, le castor gourmand et l’olivier, lieu où l’on trouve pas
moins trois restaurants distincts dont deux restaurants insaliens. 
\item Devant le prévert et le castor et pollux, lieux très fréquentés le midi
lors des pauses repas.
\item Devant le Restaurant Universitaire (RU), lieu affectionné par beaucoup
d’étudiants et pas seulement des insaliens. 
\item Un groupe a également fait un tour dans les laboratoires du CITI, où
nous retrouvons une nouvelle fois des profils distincts (professeurs et 
doctorants).
\end{itemize}

\subsection{Contenu du questionnaire}
Le questionnaire a été tourné d’une façon spécifique. Nous avons dans un 
premier temps posé des questions générales pour obtenir le profil de notre
futur utilisateur, sans préciser le pourquoi du sondage. Puis nous posions 
des questions générales sur leurs pratiques concernant les repas du midi
et procédions à des mises en situation simple. Enf, nous dévoilions le
sujet du sondage et les fonctionnalités que nous envisagions de
développer pour savoir si oui ou non l’utilisateur y voyer un intérêt ou non. 
Évidement, nous étions ouverts à toutes les idées qu’il pouvait avoir
afin de découvrir éventuellement de nouvelles fonctionnalités. \\
Voici quelques exemples de questions présentes dans le questionnaire. 
Sachant qu’il est évident que les questions différaient d’un interview à 
l’autre en fonction de l’implication de l’interviewé. 
\begin{itemize}
\item En quelle année d’étude es tu ? Es tu Insalien ?
\item Où manges tu les midis ?
\item Pourquoi ne manges tu pas ailleurs ? (essayer de forcer les points
comme : le temps d’attente, le prix, la facilité de manger avec des amis,
la proximité, la qualité, la variété des menus, le menu du jour)
\item Où manges tu si tu as moins d’une heure pour manger ?
\item Si tu avais du qu’un autre restaurant avait un temps d’attente inférieur, 
cela aurait il influencer ton choix ? 
\item Si tu arrives dans le restaurant et que le menu ne te plaît pas,
que fais tu ?
\item Si tu avais su le menu au préalable, aurai tu changer ton choix ?
\item Serai tu intéresser par des notifications sur ton mobile pour te dire 
quel restaurant correspond le mieux à tes critères ? (sachant que des 
critères tels que le prix, le temps d’attente, la proximité le menu ou autre
pourrait être pris en compte)
\item Quelles fonctionnalités supplémentaires t’intéresserai d’avoir sur 
une application pour ton smartphone ?
\item Serais tu intéressé par recharger ton forfait sur ton smartphone ?
\end{itemize}

\subsection{Échantillon de la population}
Suite a l’élaboration de ces questionnaires, nous avons recensés 
les résultats dans un tableur et discuter des différentes réponses en 
fonction du profil de notre population. Nous avons obtenu 27 réponses. 
Nous avons globalement beaucoup d’insaliens mais équitablement réparti
entre le premier cycle et le second cycle. Parmi ces étudiants, deux
étudiants d’échange sont représentés, mais avec des réponses bien
différentes, distinguant donc très clairement un nouveau profil utilisateur. 
Nous avons également un étudiant de l’université Lyon 1. Nous avons aussi
un doctorant et deux professeurs du département TC. 

\subsection{Résultats obtenus}
Dans la quasi totalité des cas, à l’exception des gens moins pressés 
pour des raisons d'horaires plus flexibles, le facteur temps est très important.
Sans cesse la réponse est : j’irai au plus proche, au plus rapide. Cela
n’est pas choquant quand on sait que à quelques minutes prêt la file
d’attente peut passer de quelques secondes d’attente à plusieurs
dizaines de minutes. Pour beaucoup d’étudiant, l’argument majeur était que 
le temps entre midi et 14 heures permet de faire de nombreuses choses : 
activités associative, travailler sur un devoir à rendre ou une IE approchante,
travail en projet, réunion de travail, sieste. Les raisons ne manquent pas 
pour expliquer que tout le monde veut manger rapidement. Certains profils
se distinguent tout de même. A l’image d’une étudiante végétarienne qui 
est très intéresser par la qualité de la nourriture et si elle trouvera à manger 
dans le restaurant qu’elle fréquente. Un autre exemple de profil distinct se
retrouve auprès de nos deux étudiants d’échange interviewés. Ceux ci
aimeraient savoir la localisation des restaurants et éventuellement pouvoir
s’y rendre. Ne connaissant pas les différents restaurants, les différents prix
pratiqués ou autre, ces étudiants se retrouvent parfois un peu perdu au
début de l’année et suivent finalement leur groupe ou leurs amis.

Quand nous discutions des fonctionnalités propres de l’application, certains
aspects se sont dégagées : 
\begin{itemize}
\item Les étudiants veulent connaître le temps d’attente en temps réel afin de
faire leur choix rapidement en sortant de cours. 
\item Aucun n’a souhaité avoir un contenu social dans cette application qui 
permettrait de manger avec ses amis, ceux qui essayent de coordonner leur
repas ne le font que rarement le midi (à l’exception des collègues d’une 
même classe) et préfèrent donc se rencontrer le soir. Ceux qui se coordonnent
le midi utilisent des applications déjà existantes telles que les SMS,
WhatsApp, Messenger, ou autre application de communication.
\item Certains ont demandé de pouvoir voir le menu sur demande avec
éventuellement une image des plats. Cela est utile pour les étudiants d’échange
qui parfois ne comprennent pas ce qui se retrouvera dans leur assiette.
\item Les étudiants sont pour l’envoie d’une notification qui leur dirait quel
restaurant correspond le mieux à leur attente pour la journée. 
\item Un affichage du prix est demandé par certains, bien que cela n’est que
peu d’importance pour d’autre étant donné que beaucoup de restaurants coûtent
globalement la même chose : un étudiant paye le même prix au Grillon, à l’Olivier,
au Prévert et Castor et Pollux. Les prix du Castor Gourmand sont variables mais
restent légèrement supérieurs au prix du RU qui arrivent en tête sur ce critère.
\end{itemize}

La partie ultérieure présentant la conception contiendra les différents écrans
composants l'application. Nous ferons des références vers cette partie afin
de justifier nos choix conceptuels.